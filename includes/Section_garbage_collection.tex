\section{Die Garbage Collection}

Garbage Collection ist ein, im Zusammenhang mit dynamischen Sprachen, weit verbreiteter Begriff, der
einen Mechanismus bezeichnet welcher für die Freigabe alloziierter Ressourcen im laufenden Programm, zuständig
ist.

In \projectname wird diese mit Hilfe der, von Boost bereitgestellten, !shared_ptr! Klasse umgesetzt.
Die Hauptimplementierung liegt hier in der Klasse !lisp::environment!, die eine Hashtable mit allen
im Programm verwendeten Symbolen hält. Symbole können auch nur über diese Klasse alloziiert werden. Der
Konstruktor der !symbol! Klasse wurde zu diesem Zweck als !private! deklariert.
Wenn nun ein Symbol über die !lisp::environment! Klasse angefordert wird, so alloziiert diese
Speicher für solch ein Symbol, speichert einen Zeiger darauf in der verwalteten Hashtable
und umschließt sie letztendlich mit einem !shared_ptr! um diesen dann an den Aufruf zurückzugeben.

Der Vorteil der genannten !shared_ptr! besteht nun darin, dass diese einen Referenzzähler unterhalten,
der die Anzahl an Kopien, dieses Zeiger, im Speicher zählt. Erreicht dieser Zähler den Wert 0, so
wird die !lisp::environment! Klasse daüber informiert. Selbige wägt dann ab ob das Symbol weiterhin benötigt
wird. Falls nicht wird es aus der gehaltenen Hashtable gelöscht.

Dieses Vorgehen unterscheidet sich grundlegend zu der Implementierung !common-lisp!. Während \projectname
dadurch eine dynamische Bindung einsetzt, hält sich !common-lisp! streng an eine syntaktische Bindung.
Die dynamische Bindung kommt hier eher der !Emacs-Lisp!-Implementierung nahe, die diese ebenfalls verfolgt.
