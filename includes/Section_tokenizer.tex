\section{Skriptverarbeitung}
%mir fällt beim besten willen kein gescheiter Titel ein
%vielleicht Ausführungsschritte?
%oder Laufzeit[irgendwas]?

\subsection{Der Tokenizer}
\label{sec:tokenizer}

Ein wichtiger zur Übersetzung von, durch Menschenhand geschriebenen, Quellcode
in maschinenverarbeitbare Speicherobjekte, ist der Tokenizer. Er sorgt für die
syntaktische Analyse von Quellcode.

Dazu wird ein Eingabestrom gelesen und sequentiell abgearbeitet und überprüft
um diese als syntaktische Elemente der Sprache zu erfassen. Tritt dabei ein
unbekanntes oder in einem syntaktischen Zusammenhang unpassendes Zeichen
auf, wird der Einlesevorgang sofort gestoppt und mit einem Fehler beendet.

Die syntaktischen Elemente des Lisp-Dialekts, den Lisp-Interaction erkennt,
sind folgende:

%Nicht nötig, aber perfekt um gelerntes zu präsentieren…
\begin{itemize}
\item{Klammern}
\item{Strings}
\item{Zahlen}
\item{Symbole}
\item{Konstanten}
\end{itemize}

Wie oben beschrieben \emph{erkennt} der Tokenizer diese Elemente lediglich,
kann deren semantische Aussagekraft nicht verarbeiten.
Für diese Aufgabe ist der Interpreter. Tatsächlich ist der Tokenizer nämlich
lediglich ein \emph{Werkzeug}, das vom Interpreter aufgerufen wird und somit
die Eingabemethode abstrahiert.

\begin{lstlisting}[caption={Tokenizer}, label=lst:tokenizer]

    enum token {
        END,
        LEFT_PARENTHESIS,
        RIGHT_PARENTHESIS,
        STRING,
        SYMBOL,
        NUMBER,
        DOT,
        QUOTE
    };

    /**
       Parses an input stream using an iterator and cuts it into
       tokens which can be consumed and processed by the interpreter
       or compiler.
    */
    template <class T>
    class tokenizer
    {
    public:
        tokenizer(T& iterator, const T& end = T())
            : m_iterator(iterator),
              m_end(end),
              m_line(1),
              m_current_token(END)
            {
            }

        token next_token()
            {
                m_cache.clear();

                if(m_iterator == m_end)
                    return END;

                char last = '\0';

                for(; m_iterator != m_end; ++m_iterator) {
                    switch(*m_iterator) {
                    case '(':
                        m_cache += *m_iterator;
                        ++m_iterator;
                        return (m_current_token = LEFT_PARENTHESIS);
                    case ')':
                        m_cache += *m_iterator;
                        ++m_iterator;
                        return (m_current_token = RIGHT_PARENTHESIS);
                    case '"':
                        parse_string();
                        return (m_current_token = STRING);
                    case '.':
                        m_cache += *m_iterator;
                        ++m_iterator;
                        return (m_current_token = DOT);
                    case '\'':
                        m_cache += *m_iterator;
                        ++m_iterator;
                        return (m_current_token = QUOTE);
                    case '\n':
                        ++m_line;
                    case ' ':
                    case '\t':
                        last = *m_iterator;
                        break;
		    case ';':
			for(; m_iterator != m_end && *m_iterator != '\n'; ++m_iterator) {
			    // Yeah, I know I could have put it in the loop-head ...
			    if(*m_iterator == '\n')
				break;
			}

			break;
                    default:
                        return parse_symbol_or_number();
                    }
                }

                if(m_iterator == m_end &&
                   (last == ' ' ||
                   last == '\t' ||
                   last == '\n'))
                    return END;

                error("unexpected end of file");

                assert(false);
            }
}

#endif  // LISP_TOKENIZER_HPP

\end{lstlisting}

%hier vielleicht noch was zum Interpreter…
\subsection{Der Interpreter}

Der Interpreter ist der Baustein, der quasi den Tokenizer umschließt.
Er ist für die semantische Analyse des Quellcodes und die Verarbeitung in
leicht handhabbare Speicherobjekte zuständig. Dazu wird der Tokenizer so oft
aufgerufen, bis ein semantisches Element komplett erkannt wurde.

Ein semantisches Element kann eines der folgenden Elemente sein:

\begin{itemize}
\item{Zahl}
\item{Bruch (Wird anders als Zahlen gehandhabt um die Genauigkeit zu behalten.)}
\item{String}
\item{Liste}
\end{itemize}

Wie die einzelnen weiterverarbeitet werden, wird durch den Nutzer gesteuert.
Es werden aber bereits Lisp-typische Verarbeitungsmethoden angeboten.
Zum Beispiel die Verarbeitung einer Liste als Funktionsaufruf.

\begin{lstlisting}[caption={Interpreter}, label=lst:interpreter]

        template <typename T>
        object_ptr_t compile_expr(environment* env, tokenizer<T>& tok)
        {
            token lisp_token = tok.current_token();
	    // logging::log(logging::DEBUG) << tok.value() << std::endl;

            switch(lisp_token) {
            case LEFT_PARENTHESIS:
                return compile_list(env, tok);
            case SYMBOL:
                if(tok.value() == "nil")
                    return nil();
                else if(tok.value() == "t")
                    return t();

                return object_ptr_t(new symbol_ref(tok.value()));
            case STRING:
                return object_ptr_t(new string(tok.value()));
            case NUMBER:
            {
                if(tok.value().find('.') != std::string::npos) {
                    double dnum = from_string<double>(tok.value());

                    number_ptr_t num = number_ptr_t(new number(dnum));

                    return num;
                }
                else if(tok.value().find('/') != std::string::npos) {
                    size_t pos = tok.value().find('/');
                    int nominator = from_string<int>(tok.value().substr(0, pos));
                    int denominator = from_string<int>(tok.value().substr(pos+1, std::string::npos));

                    number_ptr_t num = number_ptr_t(new number(nominator, denominator));

                    return num;
                }
                else {
                    long long lnum = from_string<long long>(tok.value());

                    number_ptr_t num = number_ptr_t(new number(lnum));

                    return num;
                }
            }
            case QUOTE:
                tok.next_token();
                return object_ptr_t(new quote(compile_expr(env, tok)));
            default:
                throw parse_error("unexpected token: " + tok.value(),
                                  tok.line());
            }
        }

\end{lstlisting}
