%---------------------------------------------------------------------------
%  nach einer Vorlage von Guido de Melo,
%  modifiziert für den Gebrauch im Kurs
%  Wissenschaftliche Methodik und Latex
%---------------------------------------------------------------------------

%---------------------------------------------------------------------------
%  Koma-Skript Optionen
%---------------------------------------------------------------------------
\documentclass[
    final,      % draft (keine Graphiken, overful boxes anzeigen)
    a4paper,    % alle Formate von a0paper bis a8paer
    12pt,       % Hauptschrift,9pt,10pt,11pt,12pt, für andere package "extsizes"
    titlepage,  % <-> notitlepage
    portrait,   % <-> landscape
    abstracton, % <-> abstractoff, mit \abstractname Bezeichnung festlegen
    %twocolumn,
    headings=normal, % small,normal,big
    %open=right, %any,left 
]{scrartcl} %scrartcl,scrbook,scrreprt,scrartcl,scrlttr2 

%---------------------------------------------------------------------------
%  Spracheinstellungen
%---------------------------------------------------------------------------

\usepackage[utf8]{inputenc}
\usepackage[T1]{fontenc}
\usepackage[english,ngerman]{babel}
%Trennungsregeln über
\hyphenation{Sil-ben-trenn-ung}

%---------------------------------------------------------------------------
%  Kopf- und Fußzeile
%---------------------------------------------------------------------------

% Gestaltung von Kopf- und Fußzeile über koma-Script
\usepackage[automark]{scrpage2}
	\ifoot[]{}   % [scrplain-Stil]{scrheadings-Stil}
	\cfoot[\pagemark]{}   % hier sollen der Dokumentenkopf mittig
	\ofoot[]{}   % in plain und headings leer sein
	\ihead[]{}   % leere Argumente setzen alle eventuell gesetzten Werte zurück
	\chead[]{\headmark}  % Kolumnen innen
	\ohead[]{\pagemark}  % Seitenzahlen außen
\pagestyle{scrheadings} % Aktivierung des gerade definierten Stils
                        % wichtig ist noch scrplain für Seiten mit \chapter etc.

%---------------------------------------------------------------------------
%  Seitenlayout
%---------------------------------------------------------------------------

\usepackage{geometry}
\geometry{%
	a4paper,
	top=3.5cm,
	left=3.5cm,
	right=4.5cm,
	bottom=4.5cm,
        %columnsep=1em,   % Abstände zwischen Spalten und Text
        %footnotesep=,    % Fußnoten und Text
        %marginparsep=,   % Marginalien und Text
        %headsep=,        % Kopfzeile und Text
        %footskip=,       % Fußzeile und Text
        %showframe,       % zeigt den Seitenspiegel an
	pdftex,           % Verwendung der PdfTex-Engine
}
%\usepackage{layout}     % zeigt das Layout mit allen Variabeln an über \layout

%---------------------------------------------------------------------------
%  Schriften
%---------------------------------------------------------------------------
\usepackage{mathptmx}
\usepackage[scaled]{helvet}

%\usepackage{textcomp}  % Ergänzende Zeichen, z. B. Euro und
%\usepackage{pifont}    % Zapf-Dingbats, bietet die Kommandos \ding{'64}
                        % oder \begin{dinglist}{'64}\ldots\end{dinglist}

\usepackage{shortvrb}  
\MakeShortVerb{\!}
%\DeleteShortVerb{\!}
                        % vereinfachte verbatim-Umgebung, wichtig für viel
                        % Quelltext

%---------------------------------------------------------------------------
%  Bibliographie
%---------------------------------------------------------------------------

\usepackage[style=numeric,  % Autor und Jahr, z. B. in Physik
            hyperref=true, % verlinke Bibliographie
            firstinits=true,   % Vorname zu Initialen kürzen
            isbn=false,        % keine ISBN in Bibliographie übernehmen
            ]{biblatex}
\bibliography{includes/literatur}       % Gibt den Ort der Bibliographiedatei an
%\usepackage[babel]{csquotes} % wird von Biblatex gern gesehen

%---------------------------------------------------------------------------
%  Graphiken
%---------------------------------------------------------------------------

\usepackage{graphicx}

%---------------------------------------------------------------------------
%  Tabellen
%---------------------------------------------------------------------------

\usepackage{booktabs}   % für schöne Tabellen
\usepackage{threeparttable}
%\usepackage{tabularx}  % für komplexe Tabellen

%---------------------------------------------------------------------------
%  Koma-Tuning
%---------------------------------------------------------------------------

%\setkomafont{%
%	sectioning             % alle Gliederungsüberschriften
%	caption
%	captionlabel
%	chapter
%	descriptionlabel
%	dictum
%	dictumauthor
%	dictumtext
%	footnote
%	footnotelabel
%	footnotereference      % Referenzierung der Fußnotenmarke im Text
%	minisec
%	pagefoot               % Kopf und Fuß einer Seite
%	pagehead               % Kopf und Fuß einer Seite
%	pagenumber             % Seitenzahl im Kopf oder Fuß der Seite
%	paragraph
%	part
%	partnumber             % Zeile mit der Nummer des Teils in Überschrift der Ebene \part
%	section
%	subparagraph
%	subsection
%	subsubsection
%	title
%}{% Hier der gewünschte Befehl,
		%	\normalfont, \rmfamily, \sffamily, \ttfamily,
		%	\mdseries, \bfseries, \upshape, \itshape, 
		%	\slshape, \scshape sowie Größenbefehle 
%		%	\Huge, \huge, \LARGE 
%}

%---------------------------------------------------------------------------
%  Inhalt- und Gliederung
%---------------------------------------------------------------------------

\addtokomafont{sectioning}{\rmfamily}          % behält alle gesetzten Attribute 
                                               % und verändert die Überschrift zu Serifen
\setcounter{tocdepth}{3}          % Aufgliederung des Inhaltsverzeichnisses
\setcounter{secnumdepth}{3}       % schaltet Nummering in Toc für alle Ebenen aus
                                  % bestimmen den Detailgrad des
                                  % Inhaltsverzeichnisses, drei Ebenen reichen

%---------------------------------------------------------------------------
%  Farben
%---------------------------------------------------------------------------

\usepackage[dvipsnames]{xcolor} % für sehr schöne Farben in Latex
%\newcommand*{\defaultcolor}{\color{black}}         % stellt die Standardfarbe ein

\usepackage{listings}
\lstset{
	basicstyle=\ttfamily,
	commentstyle=\color{gray},
%	keywordstyle=\textcolor{},
	stringstyle=\ttfamily,
    %	showstringspace=true,
    breaklines=true,
	numbers=left,
	numberstyle=\tiny,
	numbersep=5pt,
	language=C++}

%---------------------------------------------------------------------------
%  Seiten- und Absatzumbruch
%---------------------------------------------------------------------------

\clubpenalty10000   % keine Schusterjungen
\widowpenalty10000  % keine Hurenkinder
\parindent0pt   % kein Einzug
\flushbottom    % erzwinge gleich volle Seiten

%---------------------------------------------------------------------------
%  Für wissenschaftliche Arbeiten
%---------------------------------------------------------------------------

\newcommand{\fullname}{Vorname Nachname}
\newcommand{\email}{vorname.nachname@uni-ulm.de}
\newcommand{\titel}{Titel der Arbeit}
\newcommand{\jahr}{2009}
\newcommand{\matnr}{123456}
\newcommand{\gutachterA}{Prof.\ Dr.\ Streng Geheim}
\newcommand{\gutachterB}{Prof.\ Dr.\ Un Leserlich}
\newcommand{\betreuer}{Betreuername} % hier richtige Fakultät auswählen
\newcommand{\fakultaet}{Ingenieurwissenschaften\\und Informatik}
%\newcommand{\fakultaet}{Mathematik und\\Wirtschaftswissenschaften}
%\newcommand{\fakultaet}{Naturwissenschaften}
%\newcommand{\fakultaet}{Medizin}
% nun noch unten das Institut einsetzen
\newcommand{\institut}{Institut für Irgendetwas}
\newcommand{\projectname}{\emph{Lisp-Interaction}}

\usepackage{setspace}   % \onehalfspacing, \singlespacing, \doublespacing

%---------------------------------------------------------------------------
%  Pdf-Setup
%---------------------------------------------------------------------------

\usepackage{microtype}      % Bereitstellun des Kommandos \textls[Amount]{} für Ad-hoc-Tracking 
\usepackage[]{hyperref} 
\hypersetup{
pdftitle=\titel{},
pdfauthor=\fullname{},
pdfsubject={Abschlussarbeit},
colorlinks=false,
pdfborder=0 0 0	% keine Box um die Links!
}
