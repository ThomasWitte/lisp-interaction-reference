\section{Einführung}
\label{sec:einfuehrung}

\projectname{} ist ein minimalistischer lisp Interpreter. Er wurde entworfen um eine einfache
Integration von in C++ zu ermöglichen. Aus diesem Grund ist er in modernem Boost-unterstütztem
C++ geschrieben. Er zeichnet sich durch eine sehr komfortable und einfach zu nutzende API aus.
Ein weiterer Vorteil ist die einfache Intergrierbarkeit in C++-Projekte (Abschnitt \ref{sec:environment}).

Diese Referenz kann nur einige zentrale Punkte beleuchten und ist deshalb keinesfalls vollständig. Nähere Informationen zu Boost bietet die Boost-Do\-ku\-men\-ta\-tion auf \cite{smart_ptr_online}. Eine Einführung in Lisp bietet \cite{graham_ansi_1995}. Da \projectname{} CMake für den Buildprozess nutzt beantwortet die CMake-Dokumentation auf \cite{cmake_online} den Übersetzungsvorgang betreffende Fragen.

Da sich \projectname{} aktuell noch in Entwicklung befindet bezieht sich --- wenn nicht anders angegeben --- diese Referenz auf den Masterbranch vom 11. Mai 2011.
Das Projekt, sowie die jeweils aktuellste Version dieser Referenz stehen auf github.com zum Download bereit \cite{lisp_interaction_online}\cite{lisp_interaction_ref_online}. %Link/Quelle?!

\subsection{Allgemeiner Aufbau}

%Klassendiagramm?
%Grobe Ausführungsschritte
%Verteilung der Komponenten auf Quelldateien