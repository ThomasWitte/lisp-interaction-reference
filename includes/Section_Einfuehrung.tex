\section{Einführung}
\label{sec:einfuehrung}

\projectname{} ist ein minimalistischer lisp Interpreter. Er wurde entworfen um eine einfache
Integration von in C++ zu ermöglichen. Aus diesem Grund ist er in modernem Boost-unterstütztem
C++ geschrieben. Er zeichnet sich durch eine sehr komfortable und einfach zu nutzende API aus.
Ein weiterer Vorteil ist die einfache Intergrierbarkeit in C++-Projekte (Abschnitt \ref{sec:environment}).

Diese Referenz kann nur einige zentrale Punkte beleuchten und ist deshalb keinesfalls vollständig. Nähere Informationen zu Boost bietet die Boost-Do\-ku\-men\-ta\-tion auf \cite{smart_ptr_online}. Eine Einführung in Lisp bietet \cite{graham_ansi_1995}. Da \projectname{} CMake für den Buildprozess nutzt beantwortet die CMake-Dokumentation auf \cite{cmake_online} den Übersetzungsvorgang betreffende Fragen.

Da sich \projectname{} aktuell noch in Entwicklung befindet bezieht sich --- wenn nicht anders angegeben --- diese Referenz auf den Masterbranch vom 11. Mai 2011.
Das Projekt, sowie die jeweils aktuellste Version dieser Referenz stehen auf github.com zum Download bereit (\cite{lisp_interaction_online}\cite{lisp_interaction_ref_online}). %Link/Quelle?!

\subsection{Allgemeiner Aufbau}


\projectname{} implementiert keinen festen Lispdialekt. Die Zahlenrepräsentation ist mit Ausnahme komplexer Zahlen an Common Lisp angelehnt (\cite[S. 143ff]{graham_ansi_1995}, Abschnitt \ref{sec:numbers}), setzt aber im Gegensatz dazu, wie einige ältere Lispimplementierungen oder Emacs, dynamische Bindung ein, sucht also globale Variablen in der zum Aufrufzeitpunkt umgebenden Umgebung anstatt in der zum Deklarationszaitpunkt umgebenden (\cite[36, 63]{wilhelm_uebersetzerbau_2007}\cite[112]{graham_ansi_1995}). Listing \ref{lst:bindung} veranschaulicht dynamische und statische Bindung anhand eines Beispiels.
%habs jetzt einfach mal hier kurz reingeschrieben, vielleicht passts ja woanders besser.

\begin{lstlisting}[caption={dynamische Bindung}, label=lst:bindung, language=Lisp]
(defun test (x)
    (defun printx ()
        (print x))) ;princ in Common Lisp
(test "statisch")
;Bei statischer Bindung wird das x zum Deklarationszeitpunkt gebunden, also "statisch"

(defun test2 (x)
    (printx))
(test2 "dynamisch")
;Dynamische Bindung verwendet das x zum Aufrufzeitpunkt, also "dynamisch"
\end{lstlisting}

Die Parameterübergabe erfolgt per Value, alle Parameter einer Funktion werden also vor Ausführung der Funktion evaluiert, egal ob sie im Funktionsrumpf verwendet werden oder nicht. Lazy Evaluation, d.h. die verzögerte Auswertung von Argumenten nach Bedarf (\cite[64]{wilhelm_uebersetzerbau_2007}), wie in Haskell oder Scheme wird gegenwärtig nicht von \projectname{} unterstützt.

%Klassendiagramm?
%Grobe Ausführungsschritte
%Verteilung der Komponenten auf Quelldateien